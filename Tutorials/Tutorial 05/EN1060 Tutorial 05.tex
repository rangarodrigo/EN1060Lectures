% --------------------------------------------------------------
% This is all preamble stuff that you don't have to worry about.
% Head down to where it says "Start here"
% --------------------------------------------------------------

\documentclass[11pt]{article}
\usepackage[margin=1in]{geometry}
\usepackage{amsmath,amsthm,amssymb}
%\usepackage{multicol}
\usepackage{graphicx}
%\usepackage{fixltx2e}
%\usepackage{amsmath}

\usepackage{tikz}
\usepackage{pgfplots}
\usepackage{fourier}
\usepackage[inline]{enumitem}
\usepackage{subcaption}



%\everymath{\displaystyle}

\newcommand\ft{Fourier transform}
\newcommand\ift{inverse Fourier transform}
\newcommand\ftr{Fourier transform representation}
\newcommand\fs{Fourier series}
\newcommand\fsr{Fourier series representation}
\newcommand\xt{$x(t)$}
\newcommand\xo{$X(j\omega)$}
\newcommand\dtfs{discrete-time Fourier series}

\title{\Large Department of Electronic and Telecommunication Engineering\\University of Moratuwa\\Sri Lanka\\{\LARGE \bf \textsc{EN1060 Signals and Systems: Tutorial 05 \footnote{All the questions are from Oppenheim \emph{et al.} chapter 4.}}}}

\date{\vspace{-0.2in}\today}


\newcommand{\N}{\mathbb{N}}
\newcommand{\Z}{\mathbb{Z}}

\begin{document}



\maketitle
\noindent \tikz \draw (0,0) -- (\textwidth,0);

\begin{enumerate}
    % Q01 Oppenheim et al. 3.2
    \item A discrete-time periodic signal $x[n]$ is real valued and has a fundamental period $N = 5$. The nonzero Fourier series coefficients for $x[n]$ are
    \begin{equation*}
        a_0 = 1, a_2 = a^\ast_2 = e^{j\pi/4}, a_4 = a^\ast_4 = 2e^{j\pi/3}.
    \end{equation*}
    Express $x[n]$ in the form
    \begin{equation*}
        x[n] = A_0 + \sum_{k=1}^{\infty}A_k \sin(\omega_k n + \phi_k)
    \end{equation*}

    % Q02 Oppenheim et al. 3.9
    \item Use the \dtfs~analysis equation to evaluate the numerical values of one period of the Fourier series coefficients of the periodic signal
    \begin{equation*}
        x[n] = \sum_{m=-\infty}^{\infty}\left\{ 4\delta[n-4m] + 8\delta[n-1-4m]\right\}
    \end{equation*}

    % Q03 Oppenheim et al. 3.10
    \item Let $x[n]$ be a real and odd periodic signal with period $N = 7$ and Fourier coefficients $a_k$. Given that
    \begin{equation*}
        a_{15} = j, a_{16} = 2j, a_{17} = 3j,
    \end{equation*}
    determine the values of $a_0$, $a_1$, $a_{-2}$,  and $a_{-3}$.

    % Q03 Oppenheim et al. 3.11
    \item Suppose we are given the following information about a signal $x[n]$:
    \begin{enumerate}
        \item $x[n]$ is a real and even signal.
        \item $x[n]$ has period $N = 10$ and Fourier coefficients $a_k$.
        \item $a_{11} = 5$.
        \item $\frac{1}{10}\sum_{n=0}^{9}|x[n]|^2 = 50.$
    \end{enumerate}
    Show that $x[n] = A \cos(Bn +C)$, and specify numerical values for the constants $A$, $B$, and $C$.

    % Q03 Oppenheim et al. 3.12, 3.13, 3.14, 3.16, 3.31, 3.36, 3.48

    % Q03 Oppenheim et al. 5.1 5.3 5.4 5.6 5.7 5.8 5.9 5.12 5.13 5.14 5.19 5.21 5.22 5.26 5.29 5.35 5.50


\end{enumerate}

\end{document} 